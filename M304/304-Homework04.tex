\documentclass[11pt]{article}		% The percent symbol in your code starts a comment.  The comment ends at the next linebreak.

\usepackage[english]{babel} 		% Packages add functionality and style conventions to your documents. Don't edit this section!
\usepackage{fullpage}				% Eliminates wasted space
\usepackage[utf8]{inputenc}			% Necessary for character encoding
\usepackage{amsmath, amssymb,amsthm}% Required math packages
\usepackage{graphicx}				% For handling graphics
\usepackage[colorinlistoftodos]{todonotes}	% For the fancy "todo" stuff
\usepackage{hyperref}				% For clickable links in the final PDF
\usepackage{titling}				% To take less space at the top of the page with the title
\setlength{\droptitle}{-2cm}
\pretitle{\begin{flushright}\Large\scshape}
\posttitle{\par\end{flushright}}
\preauthor{\begin{flushright}\large\scshape}
\postauthor{\par\end{flushright}}
\predate{\begin{flushright}\large\scshape}
\postdate{\par\end{flushright}}


% Type `\C' for the complex numbers, `\H' for the quarternions, etc.
\def\C{{\mathbb C}}
\def\H{{\mathbb H}}
\def\Z{{\mathbb Z}}
\def\Q{{\mathbb Q}}
\def\R{{\mathbb R}}
\def\N{{\mathbb N}}


%\Alpha{homeworkresults}

\theoremstyle{definition}
\newtheorem{theorem}{Theorem}
\renewcommand*{\thetheorem}{\Alph{theorem}}
\setcounter{theorem}{6}
\newtheorem{lemma}[theorem]{Lemma}
\newtheorem{prop}[theorem]{Proposition}
\newtheorem{claim}[theorem]{Claim}
\newtheorem{example}[theorem]{Example}
\newtheorem{conj}[theorem]{Conjecture}




\title{Math 304 Homework 4}

\author{Your name goes here}

\date{Due September 28, 2018}

\begin{document}
\maketitle


Let $K = \{2^r \ : \ r\in \R\}$, and define binary operations on $K$ as follows:
\begin{itemize}
	\item Addition: $2^r \oplus 2^s := 2^{r+s}$
	\item Multiplication: $2^r \otimes 2^s := 2^{rs}$
\end{itemize}

Prove or disprove the following conjecture.

\begin{conj}
	With the operations $\oplus$ and $\otimes$, $K$ is a field. [Note that your solution should include either a careful verification of each of the field axioms, or a specific counterexample of a field axiom that is not satisfied.]
\end{conj}

\medskip

\noindent\textit{Solution:}				 \hfill $\lozenge$


\bigskip

A \textit{Boolean ring} $R$ is a ring in which $x^2 = x$ for all $x\in R$.

\begin{theorem}
	Let $R$ be a Boolean ring.
	Then: 
	\begin{itemize}
		\item For all $r\in R$, $r + r = 0_R$. [Hint: Square a convenient element of $R$.]
		\item $R$ is commutative. [Hint: Square a different convenient element of $R$, then use the previous part.]
	\end{itemize}
\end{theorem}

\begin{proof}

\end{proof}


\end{document}