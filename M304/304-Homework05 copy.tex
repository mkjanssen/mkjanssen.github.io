\documentclass[11pt]{article}		% The percent symbol in your code starts a comment.  The comment ends at the next linebreak.

\usepackage[english]{babel} 		% Packages add functionality and style conventions to your documents. Don't edit this section!
\usepackage{fullpage}				% Eliminates wasted space
\usepackage[utf8]{inputenc}			% Necessary for character encoding
\usepackage{amsmath, amssymb,amsthm}% Required math packages
\usepackage{graphicx}				% For handling graphics
\usepackage[colorinlistoftodos]{todonotes}	% For the fancy "todo" stuff
\usepackage{hyperref}				% For clickable links in the final PDF
\usepackage{titling}				% To take less space at the top of the page with the title
\setlength{\droptitle}{-2cm}
\pretitle{\begin{flushright}\Large\scshape}
\posttitle{\par\end{flushright}}
\preauthor{\begin{flushright}\large\scshape}
\postauthor{\par\end{flushright}}
\predate{\begin{flushright}\large\scshape}
\postdate{\par\end{flushright}}


% Type `\C' for the complex numbers, `\H' for the quarternions, etc.
\def\C{{\mathbb C}}
\def\H{{\mathbb H}}
\def\Z{{\mathbb Z}}
\def\Q{{\mathbb Q}}
\def\R{{\mathbb R}}
\def\N{{\mathbb N}}


%\Alpha{homeworkresults}

\theoremstyle{definition}
\newtheorem{theorem}{Theorem}
\renewcommand*{\thetheorem}{\Alph{theorem}}
\setcounter{theorem}{10}
\newtheorem{lemma}[theorem]{Lemma}
\newtheorem{prop}[theorem]{Proposition}
\newtheorem{claim}[theorem]{Claim}
\newtheorem{example}[theorem]{Example}
\newtheorem{conj}[theorem]{Conjecture}
\newcommand{\ideal}[1]{\left\langle\, #1 \,\right\rangle}
	\def\set#1{\left\{ {#1} \right\}}
%	\def\setof#1#2{{\left\{#1\,\middle\rvert\,#2\right\}}}
	\def\setof#1#2{{\left\{#1\,\colon\,#2\right\}}}


\title{Math 304 Homework 6}

\author{Your name goes here}

\date{Due October 26, 2018}

\begin{document}
\maketitle


\begin{theorem}
	Let $R$ be a commutative ring with identity, and $I,J\subseteq R$ ideals.
	
	\begin{enumerate}
		\item The \emph{sum} $I+J$ of $I$ and $J$ is an ideal of $R$, where
		\[
			I+J := \setof{x+y}{x\in I, \ y\in J}.
		\]
		\item The \emph{product} $IJ$ of $I$ and $J$ is an ideal of $R$, where
		\[
			IJ := \setof{x_1 y_1 + x_2 y_2 + \cdots + x_n y_n}{n\ge 0, \ x\in I, \ y\in J}.
		\]
		\item The $n$th power of $I$, $I^n$ is an ideal of $R$, where
		\[
			I^n = \underbrace{I\cdot I \cdots I}_{n\text{ times}}.
		\]
	\end{enumerate}
\end{theorem}

\begin{proof}

\end{proof}


In the following theorem, fill in the blank, and then complete the proof.

\begin{theorem}
	Let $m,n\in\Z$.
	If $a$ is \makebox[0.75in]{\hrulefill}, then $\ideal{a} = \ideal{m} + \ideal{n}$.
\end{theorem}

\begin{proof}

\end{proof}
 

\end{document}