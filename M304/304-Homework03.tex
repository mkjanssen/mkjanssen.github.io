\documentclass[11pt]{article}		% The percent symbol in your code starts a comment.  The comment ends at the next linebreak.

\usepackage[english]{babel} 		% Packages add functionality and style conventions to your documents. Don't edit this section!
\usepackage{fullpage}				% Eliminates wasted space
\usepackage[utf8]{inputenc}			% Necessary for character encoding
\usepackage{amsmath, amssymb,amsthm}% Required math packages
\usepackage{graphicx}				% For handling graphics
\usepackage[colorinlistoftodos]{todonotes}	% For the fancy "todo" stuff
\usepackage{hyperref}				% For clickable links in the final PDF
\usepackage{titling}				% To take less space at the top of the page with the title
\setlength{\droptitle}{-2cm}
\pretitle{\begin{flushright}\Large\scshape}
\posttitle{\par\end{flushright}}
\preauthor{\begin{flushright}\large\scshape}
\postauthor{\par\end{flushright}}
\predate{\begin{flushright}\large\scshape}
\postdate{\par\end{flushright}}


% Type `\C' for the complex numbers, `\H' for the quarternions, etc.
\def\C{{\mathbb C}}
\def\H{{\mathbb H}}
\def\Z{{\mathbb Z}}
\def\Q{{\mathbb Q}}
\def\R{{\mathbb R}}
\def\N{{\mathbb N}}


%\Alpha{homeworkresults}

\theoremstyle{definition}
\newtheorem{theorem}{Theorem}
\renewcommand*{\thetheorem}{\Alph{theorem}}
\setcounter{theorem}{4}
\newtheorem{lemma}[theorem]{Lemma}
\newtheorem{prop}[theorem]{Proposition}
\newtheorem{claim}[theorem]{Claim}
\newtheorem{example}[theorem]{Example}
\newtheorem{conj}[theorem]{Conjecture}




\title{Math 304 Homework 3}

\author{Your name goes here}

\date{Due September 21, 2018}

\begin{document}
\maketitle


In the following theorem, fill in the blank, and then complete the proof.

\begin{theorem}
	Let $m,n\in\N$, and let $a\in \Z$.
	Let $[a]_m$ denote the equivalence class of $a$ under $\equiv_m$, and $[a]_n$ the equivalence class of $a$ under $\equiv_n$.
	Then $[a]_m\subseteq [a]_n$ if and only if \makebox[0.75in]{\hrulefill}.
\end{theorem}

\begin{proof}

\end{proof}


\medskip


A relation $\sim$ on a set $S$ is said to be \emph{circular} provided that for all $a,b,c\in S$, if $a\sim b$ and $b\sim c$, then $c\sim a$.

\begin{theorem}
	A relation $\sim$ on a set $S$ is an equivalence relation if and only if $\sim$ is reflexive and circular.
\end{theorem}

\begin{proof}
	
\end{proof}


\end{document}