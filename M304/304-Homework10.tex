\documentclass[11pt]{article}		% The percent symbol in your code starts a comment.  The comment ends at the next linebreak.

\usepackage[english]{babel} 		% Packages add functionality and style conventions to your documents. Don't edit this section!
\usepackage{fullpage}				% Eliminates wasted space
\usepackage[utf8]{inputenc}			% Necessary for character encoding
\usepackage{amsmath, amssymb,amsthm}% Required math packages
\usepackage{graphicx}				% For handling graphics
\usepackage[colorinlistoftodos]{todonotes}	% For the fancy "todo" stuff
\usepackage{hyperref}				% For clickable links in the final PDF
\usepackage{titling}				% To take less space at the top of the page with the title
\setlength{\droptitle}{-2cm}
\pretitle{\begin{flushright}\Large\scshape}
\posttitle{\par\end{flushright}}
\preauthor{\begin{flushright}\large\scshape}
\postauthor{\par\end{flushright}}
\predate{\begin{flushright}\large\scshape}
\postdate{\par\end{flushright}}


% Type `\C' for the complex numbers, `\H' for the quarternions, etc.
\def\C{{\mathbb C}}
\def\H{{\mathbb H}}
\def\Z{{\mathbb Z}}
\def\Q{{\mathbb Q}}
\def\R{{\mathbb R}}
\def\N{{\mathbb N}}


%\Alpha{homeworkresults}

\theoremstyle{definition}
\newtheorem{theorem}{Theorem}
\renewcommand*{\thetheorem}{\Alph{theorem}}
\setcounter{theorem}{18}
\newtheorem{lemma}[theorem]{Lemma}
\newtheorem{prop}[theorem]{Proposition}
\newtheorem{claim}[theorem]{Claim}
\newtheorem{example}[theorem]{Example}
\newtheorem{exercise}[theorem]{Exercise}
\newtheorem{conj}[theorem]{Conjecture}
\newcommand{\ideal}[1]{\left\langle\, #1 \,\right\rangle}
	\def\set#1{\left\{ {#1} \right\}}
%	\def\setof#1#2{{\left\{#1\,\middle\rvert\,#2\right\}}}
	\def\setof#1#2{{\left\{#1\,\colon\,#2\right\}}}


\title{Math 304 Homework 10}

\author{Your name goes here}

\date{Due November 30, 2018}

\begin{document}
\maketitle


Let $R = \R[x,y]$ and let $f(x,y), g(x,y)\in \R[x,y]$ be non-constant polynomials.
Define the \emph{zero set} of $f$ and $g$, $Z(f,g)\subseteq \R^2$, by:
\[
	Z(f,g) := \setof{(a,b)\in \R^2}{f(a,b) = g(a,b) = 0}.
\]
Thus, for example, $Z(y-x^3, y - x) = \set{(-1,-1),(0,0),(1,1)}$, while $Z(y-x^2+5x-4)$ contains all the points on the graph of the parabola given by $y = x^2 -5x + 4$.

Define $I(Z(f,g)) = \setof{p\in R}{p(a,b) = 0 \text{ for all } (a,b)\in Z(f,g)}$.


\begin{theorem}
	The set $I(Z(f,g))$ is an ideal of $R$ containing $\ideal{f,g}$.
\end{theorem}

\begin{proof}

\end{proof}

In general, $I(Z(f,g)) \ne \ideal{f,g}$, though if we replace $\R$ by $\C$ and assume an additional technical condition on $f$ and $g$, equality does hold.




\begin{theorem}
	Let $S = \setof{\left(\begin{matrix} a & b \\ -b & a \end{matrix}\right)}{a,b\in \R}$.
	Prove that $\phi : \C \to S$ given by
	\[
		\phi(a+bi) = \left(\begin{matrix} a & b \\ -b & a \end{matrix}\right)
	\]
	is a ring isomorphism.	
\end{theorem}


\begin{proof}

\end{proof} 
 



\end{document}