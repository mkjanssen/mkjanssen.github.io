\documentclass[11pt]{article}		% The percent symbol in your code starts a comment.  The comment ends at the next linebreak.

\usepackage[english]{babel} 		% Packages add functionality and style conventions to your documents. Don't edit this section!
\usepackage{fullpage}				% Eliminates wasted space
\usepackage[utf8]{inputenc}			% Necessary for character encoding
\usepackage{amsmath, amssymb,amsthm}% Required math packages
\usepackage{graphicx}				% For handling graphics
\usepackage[colorinlistoftodos]{todonotes}	% For the fancy "todo" stuff
\usepackage{hyperref}				% For clickable links in the final PDF
\usepackage{titling}				% To take less space at the top of the page with the title
\setlength{\droptitle}{-2cm}
\pretitle{\begin{flushright}\Large\scshape}
\posttitle{\par\end{flushright}}
\preauthor{\begin{flushright}\large\scshape}
\postauthor{\par\end{flushright}}
\predate{\begin{flushright}\large\scshape}
\postdate{\par\end{flushright}}


% Type `\C' for the complex numbers, `\H' for the quarternions, etc.
\def\C{{\mathbb C}}
\def\H{{\mathbb H}}
\def\Z{{\mathbb Z}}
\def\Q{{\mathbb Q}}
\def\R{{\mathbb R}}
\def\N{{\mathbb N}}


%\Alpha{homeworkresults}

\theoremstyle{definition}
\newtheorem{theorem}{Theorem}
\renewcommand*{\thetheorem}{\Alph{theorem}}
\setcounter{theorem}{16}
\newtheorem{lemma}[theorem]{Lemma}
\newtheorem{prop}[theorem]{Proposition}
\newtheorem{claim}[theorem]{Claim}
\newtheorem{example}[theorem]{Example}
\newtheorem{exercise}[theorem]{Exercise}
\newtheorem{conj}[theorem]{Conjecture}
\newcommand{\ideal}[1]{\left\langle\, #1 \,\right\rangle}
	\def\set#1{\left\{ {#1} \right\}}
%	\def\setof#1#2{{\left\{#1\,\middle\rvert\,#2\right\}}}
	\def\setof#1#2{{\left\{#1\,\colon\,#2\right\}}}


\title{Math 304 Homework 9}

\author{Your name goes here}

\date{Due November 16, 2018}

\begin{document}
\maketitle

If every ring had the unique factorization property, life would be very boring indeed.
And in fact, the failure of certain rings in algebraic number theory to have the unique factorization property played a role in several failed attempts to prove Fermat's Last Theorem, which says that there are no nontrivial integer solutions to the equation $x^n + y^n = z^n$ if $n \ge 3$.

In 1847, Gabriel Lam\'{e} claimed he had complete solved the problem. His solution relied on the factorization of $x^p + y^p$, where $p$ is an odd prime, as
\[
	x^p + y^p = (x+y)(x+\zeta y) \cdots (x+\zeta^{p-1}y),
\]
where $\zeta = e^{2\pi i/p}$ is a primitive $p$-th root of unity in $\C$.
However, the ring 
\[
\Z[\zeta] = \setof{a_0 + a_1 \zeta + a_2 \zeta^2 + \cdots + a_{p-1} \zeta^{p-1}}{a_i\in\Z}
\]
is not a unique factorization domain.

In the following exercises, we explore factorization in a similar integral domain, $R = \Z[\sqrt{-5}] = \setof{a+b\sqrt{-5}}{a,b\in\Z}$.


\begin{lemma}
	There do not exist nonnegative integers $x, y, s, t$ such that $2 = x^2+5y^2$ or $3 = s^2 + 5 t^2$.
\end{lemma}

\begin{proof}

\end{proof}





\begin{theorem}
	In $R = \Z[\sqrt{-5}]$ the elements $2, 3, 1 + \sqrt{-5}$, and $1-\sqrt{-5}$ are irreducible.
	Moreover, $R$ is not a UFD.
	[Hint: Apply Theorem N.]
\end{theorem}


\begin{proof}

\end{proof} 
 

\end{document}