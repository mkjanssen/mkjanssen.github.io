\documentclass[11pt]{article}		% The percent symbol in your code starts a comment.  The comment ends at the next linebreak.

\usepackage[english]{babel} 		% Packages add functionality and style conventions to your documents. Don't edit this section!
\usepackage{fullpage}				% Eliminates wasted space
\usepackage[utf8]{inputenc}			% Necessary for character encoding
\usepackage{amsmath, amssymb,amsthm}% Required math packages
\usepackage{graphicx}				% For handling graphics
\usepackage[colorinlistoftodos]{todonotes}	% For the fancy "todo" stuff
\usepackage{hyperref}				% For clickable links in the final PDF
\usepackage{titling}				% To take less space at the top of the page with the title
\setlength{\droptitle}{-2cm}
\pretitle{\begin{flushright}\Large\scshape}
\posttitle{\par\end{flushright}}
\preauthor{\begin{flushright}\large\scshape}
\postauthor{\par\end{flushright}}
\predate{\begin{flushright}\large\scshape}
\postdate{\par\end{flushright}}


% Type `\C' for the complex numbers, `\H' for the quarternions, etc.
\def\C{{\mathbb C}}
\def\H{{\mathbb H}}
\def\Z{{\mathbb Z}}
\def\Q{{\mathbb Q}}
\def\R{{\mathbb R}}
\def\N{{\mathbb N}}


%\Alpha{homeworkresults}

\theoremstyle{definition}
\newtheorem{theorem}{Theorem}
\renewcommand*{\thetheorem}{\Alph{theorem}}
\setcounter{theorem}{22}
\newtheorem{lemma}[theorem]{Lemma}
\newtheorem{prop}[theorem]{Proposition}
\newtheorem{claim}[theorem]{Claim}
\newtheorem{example}[theorem]{Example}
\newtheorem{exercise}[theorem]{Exercise}
\newtheorem{conj}[theorem]{Conjecture}
\newcommand{\ideal}[1]{\left\langle\, #1 \,\right\rangle}
	\def\set#1{\left\{ {#1} \right\}}
%	\def\setof#1#2{{\left\{#1\,\middle\rvert\,#2\right\}}}
	\def\setof#1#2{{\left\{#1\,\colon\,#2\right\}}}


\title{Math 304 Extra Board Work}

\author{Your name goes here}

\date{Due December 12, 2018}

\begin{document}
\maketitle

Choose \textbf{one} of the following. If you do more than one problem, I will grade the first one I see. 

Your work will be graded out of 1 point on the scale a board work question would be graded: 1 point for a clear and correct proof, 1/2 point if it is unclear or somewhat incorrect, and 1/4 point if it is in completely the wrong direction. You are welcome to ask for help as you prepare your proof.
 
\begin{exercise}
	Let $G$ be a group and fix an element $g\in G$.
	Define $\varphi_g : G\to G$ by $\varphi(x) = g x g^{-1}$.
	Prove that $\varphi_g$ is a homomorphism, and determine whether or not it is an isomorphism. 
\end{exercise}

\begin{proof}

\end{proof}


Let $G$ be a group and $N\leq G$.
We say $N$ is a \emph{normal} subgroup if for all $a\in G$, $a N = Na$, where $aN = \setof{an}{n\in N}$ and $Na = \setof{na}{n\in N}$.
We denote this by $N\vartriangleleft G$.

\begin{theorem}
	A subgroup $H$ of $G$ is normal if and only if $x H x^{-1}\subseteq H$ for all $x\in G$.
\end{theorem}

\begin{proof}

\end{proof}


\begin{theorem}
	Let $f : G\to K$ be a homomorphism.
	Then $\ker f \vartriangleleft G$.
	[You may use without proof the fact that $f(x^{-1}) = (f(x))^{-1}$.]
\end{theorem}

\begin{proof}

\end{proof}

\begin{theorem}
	Let $G$ be a group.
	The \emph{center} of $G$ is the set
	\[
		Z(G) = \setof{g\in G}{gh=hg \forall h\in G}.
	\]
	Prove that $Z(G)\vartriangleleft G$.
\end{theorem}

\begin{proof}

\end{proof}



\end{document}