\documentclass[11pt]{article}		% The percent symbol in your code starts a comment.  The comment ends at the next linebreak.

\usepackage[english]{babel} 		% Packages add functionality and style conventions to your documents. Don't edit this section!
\usepackage{fullpage}				% Eliminates wasted space
\usepackage[utf8]{inputenc}			% Necessary for character encoding
\usepackage{amsmath, amssymb,amsthm}% Required math packages
\usepackage{graphicx}				% For handling graphics
\usepackage[colorinlistoftodos]{todonotes}	% For the fancy "todo" stuff
\usepackage{hyperref}				% For clickable links in the final PDF
\usepackage{titling}				% To take less space at the top of the page with the title
\setlength{\droptitle}{-2cm}
\pretitle{\begin{flushright}\Large\scshape}
\posttitle{\par\end{flushright}}
\preauthor{\begin{flushright}\large\scshape}
\postauthor{\par\end{flushright}}
\predate{\begin{flushright}\large\scshape}
\postdate{\par\end{flushright}}
\linespread{1.5}

% Type `\C' for the complex numbers, `\H' for the quarternions, etc.
\def\C{{\mathbb C}}
\def\H{{\mathbb H}}
\def\Z{{\mathbb Z}}
\def\Q{{\mathbb Q}}
\def\R{{\mathbb R}}
\def\N{{\mathbb N}}


%\Alpha{homeworkresults}

\theoremstyle{definition}
\newtheorem{theorem}{Theorem}
\renewcommand*{\thetheorem}{\Alph{theorem}}
\setcounter{theorem}{15}
\newtheorem{lemma}[theorem]{Lemma}
\newtheorem{prop}[theorem]{Proposition}
\newtheorem{claim}[theorem]{Claim}
\newtheorem{example}[theorem]{Example}
\newtheorem{statement}[theorem]{Statement}




\title{Math 212 Homework 6}

\author{Your name goes here}

\date{Due April 10, 2019}

\begin{document}
\maketitle

\noindent\textbf{List of collaborators:}

\noindent For Theorem \ref{thm:indexsets}, prove one of parts 1--2, and one of parts 3--4.
You will need to chase elements.

\begin{theorem}\label{thm:indexsets}
	Let $I$ be an index set and let $\mathcal{F} = \{T_i\}_{i\in I}$ be a family of sets in a universal set $U$.
	Let $X\subseteq U$.
	Then:
		\begin{enumerate}
			\item $\left(\bigcap\limits_{i\in I} T_i\right)^c = \bigcup\limits_{i\in I} T_i^c$
			\item $\left(\bigcup\limits_{i\in I} T_i\right)^c = \bigcap\limits_{i\in I} T_i^c$
			\item $X \cap \left(\bigcup\limits_{i\in I} T_i\right) = \bigcup\limits_{i\in I} (X\cap T_i)$
			\item $X \cup \left(\bigcap\limits_{i\in I} T_i\right) = \bigcap\limits_{i\in I} (X\cup T_i)$
		\end{enumerate}
\end{theorem}

\begin{proof}

\end{proof}

\noindent For Theorem \ref{thm:relations}, after you have proved that $\mathcal{R}$ is an equivalence relation, you need to state what the equivalence (relation) classes are, describe them in more familiar terms, and prove that you have found them all.

\begin{theorem}\label{thm:relations}
	The relation $\mathcal{R}$ defined on $\Z$ by $(m,n)\in \mathcal{R}$ if and only if $m+n$ is even is an equivalence relation.
	The relation classes are \makebox[0.75in]{\hrulefill}.
\end{theorem}

\begin{proof} 

\end{proof}


\noindent For Theorem \ref{thm:partialorder}, we make the following definition.

\noindent\textbf{Definition. } Let $S$ be a set and $\mathcal{R}$ a relation on $S$.
We say that $\mathcal{R}$ is a \emph{partial ordering} on $S$ if it is reflexive, transitive, and \emph{anti-symmetric}: if $x,y\in S$ and both $(x,y)\in \mathcal{R}$ and $(y,x)\in \mathcal{R}$, then $x = y$.
Furthermore, we say that two elements $x,y\in S$ are \emph{comparable} under $\mathcal{R}$ if either $(x,y)\in \mathcal{R}$ or $(y,x)\in\mathcal{R}$.
If any two elements of $S$ are comparable under $\mathcal{R}$, we say $\mathcal{R}$ is a \emph{total ordering}.

\begin{theorem}\label{thm:partialorder}
	The relation $\subseteq$ on the set $\mathcal{P}(\Z)$ is a partial ordering that is not a total ordering.
\end{theorem}

\begin{proof}

\end{proof}

\end{document}