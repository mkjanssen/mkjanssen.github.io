\documentclass[11pt]{article}		% The percent symbol in your code starts a comment.  The comment ends at the next linebreak.

\usepackage[english]{babel} 		% Packages add functionality and style conventions to your documents. Don't edit this section!
\usepackage{fullpage}				% Eliminates wasted space
\usepackage[utf8]{inputenc}			% Necessary for character encoding
\usepackage{amsmath, amssymb,amsthm}% Required math packages
\usepackage{graphicx}				% For handling graphics
\usepackage[colorinlistoftodos]{todonotes}	% For the fancy "todo" stuff
\usepackage{hyperref}				% For clickable links in the final PDF
\usepackage{titling}				% To take less space at the top of the page with the title
\setlength{\droptitle}{-2cm}
\pretitle{\begin{flushright}\Large\scshape}
\posttitle{\par\end{flushright}}
\preauthor{\begin{flushright}\large\scshape}
\postauthor{\par\end{flushright}}
\predate{\begin{flushright}\large\scshape}
\postdate{\par\end{flushright}}
\linespread{1.5}

% Type `\C' for the complex numbers, `\H' for the quarternions, etc.
\def\C{{\mathbb C}}
\def\H{{\mathbb H}}
\def\Z{{\mathbb Z}}
\def\Q{{\mathbb Q}}
\def\R{{\mathbb R}}
\def\N{{\mathbb N}}


%\Alpha{homeworkresults}

\theoremstyle{definition}
\newtheorem{theorem}{Theorem}
\renewcommand*{\thetheorem}{\Alph{theorem}}
\setcounter{theorem}{18}
\newtheorem{lemma}[theorem]{Lemma}
\newtheorem{prop}[theorem]{Proposition}
\newtheorem{claim}[theorem]{Claim}
\newtheorem{example}[theorem]{Example}
\newtheorem{statement}[theorem]{Statement}




\title{Math 212 Homework 7}

\author{Your name goes here}

\date{Due May 1, 2019}

\begin{document}
\maketitle

\noindent\textbf{List of collaborators:}



\begin{theorem}
	Let $G$ be a graph with at least two vertices.
	Then two of its vertices have the same degree. % [Hint: Pigeonhole Principle!]
\end{theorem}


\begin{proof}

\end{proof}



\begin{theorem}
	The average degree of the vertices of a tree is less than 2.
\end{theorem}

\begin{proof}

\end{proof}

\noindent \textbf{Definition.} The \emph{star graph on $n$ vertices} has one vertex adjacent to all other vertices (and no other adjacencies). 

For the following, fill in the blank and prove your theorem.

\begin{theorem}
	The star graph on $n$ vertices has \makebox[0.75in]{\hrulefill} edges. % [Hint: induction!]
\end{theorem}

\begin{proof}

\end{proof}



\end{document}